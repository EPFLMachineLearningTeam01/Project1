\documentclass[10pt,conference,compsocconf]{IEEEtran}

\usepackage{hyperref}
\usepackage{graphicx}	% For figure environment
\usepackage{authblk}

\title{Project 1 on Machine Learning team Yoor}

\author[1]{Sergei Volodin}
\author[1]{Baran Nama}
\author[1]{Name here 3}
\affil[1]{EPFL}
\affil[ ]{\textit {\{sergei.volodin,baran.nama,email3\}@epfl.ch}}

\begin{document}

\maketitle

\begin{abstract}
A classification dataset from LHC is being studied. First, the data is thoroughly explored using visual aids. Several basic Machine Learning methods are applied. Results are evaluated using cross-validation. Model overview is given and the best model is chosen.
\end{abstract}

\section{Introduction}
Claim: it is possible to use simple methods for this dataset
\begin{enumerate}
	\item What is the data (Simulation from LHC, details from physics)
	\item What are we trying to do? Get the best classification score
	\item Overview of data, diagrams of features, feature selection, feature augmentation
	\item Methods and their choice (Linear regression, logistic regression, ridge regression) because of simplicity
\end{enumerate}
\section{Models and Methods}
\begin{enumerate}
\item Least squares. Problem: missing data, overfit
\item Mean imputation. Problem: overfit, meaninglessness for some features
\item Feature binarization, add new feature 'feature missing', add squares for features for mass
\item Ridge regression using k-fold. Problem: low accuracy (?)
\item Logistic regression
\item Nearest neighbors?
\end{enumerate}
\section{Results}
Shows that accuracy is good enough meaning that model selection was good
\section{Discussion}
State that we can improve the accuracy by using non-linear classifiers?
\section{Summary}
We have shown that it is possible to detect the Higgs boson using linear methods and feature augmentation.

\end{document}
